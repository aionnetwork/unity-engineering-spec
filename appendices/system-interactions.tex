\section{System Interactions}
\label{appendix:sequence_diagrams}

\subsection{Private staking and unstaking}
Figure \ref{fig:priv_staking} describes a very simple scenario, where a Staker chooses to privatly stake their coins 

the sequence diagram for a staker to privately stake their coins (i.e. without going through the ADS or any other delegation mechanism).  The staker has to first register themselves with the StakerRegistry contract; then they could send vote, messages, voting for themselves, sending over amount to stake. When they're ready to unstake their coins, they have to send two transactions: one to perform the unvote (with the amount to unvote), and one to release the stake, after the lockout period is over. This design is due to the fact that we don't have onBlockProduced events to trigger release of coins after the 

\begin{figure}[ht]
\centering
\includegraphics{sequence-diagrams/private-staking.tikz}
\caption{Private staking interactions}
\label{fig:priv_staking}
\end{figure}
\clearpage

\subsection{Pool setup}
Figure \ref{fig:pool_setup} describes the setup of a pool by a pool operator. The function arguments are omitted for brevity. Please refer to Appendix \ref{appendix:contract} for full contract and detailed function arguments. 

First, the operator must sign up with the StakerRegistry as a staker (using a dummy coinbase address, which we will change later). Then he must sign up in the Pool registry with  the appropriate setup parameters. The setup call will deplay the coinbase contract and return it's address. The operator must now make one final calle to the staking registry to update the  
the sequence diagram for a staker to privately stake their coins (i.e. without going through the ADS or any other delegation mechanism).  The staker has to first register themselves with the StakerRegistry contract; then they could send vote, messages, voting for themselves, sending over amount to stake. When they're ready to unstake their coins, they have to send two transactions: one to perform the unvote (with the amount to unvote), and one to release the stake, after the lockout period is over. This design is due to the fact that we don't have onBlockProduced events to trigger release of coins after the 

Note that you can't addListener before registerPool() ...

\begin{figure}[ht]
\centering
\includegraphics{sequence-diagrams/pool-setup.tikz}
\caption{Pool setup}
\label{fig:pool_setup}
\end{figure}
\clearpage

\subsection{Delegate and undelegate coins}

Figure \ref{fig:delegation} shows the lifecycle events of delegation and undelegation. All interactions that the delegator does are with the PoolRegistry. For delegation, the user must send the coins to the PoolRegistry as part of the delegate(..) transaction. When delegate(..) gets called, it in-turn records the delegation and adds the stake to himself in the StakerRegistry; to the staking registry, the pool looks like one big staker, constantly adding more and more stake. 

For the undelegation, it is a two-step process. The user calls undelegate on the PoolRegistry with the address of the pool they want to undelegate from. As part of this transaction, the PoolRegistry contract turns around and calls unvoteTo(..) to remove the stake from the active stake, and have the stake be eligible to be withdrawn from the StakerRegistry by it's original owner. After the lockout period has elapsed, the user can call releaseStake on the PoolRegistry contract, which is just a wrapper around the releaseStake(..) functionality in the StakerRegistry contract. 

\begin{figure}[ht]
\centering
\includegraphics{sequence-diagrams/delegation.tikz}
\caption{Coin delegation and undelegation}
\label{fig:delegation}
\end{figure}
\clearpage

\subsection{Auto-Redelegation}

Figure \ref{fig:redelegation} demonstrates a auto-redelegation feature. There are two paths into Auto-redelegation. One while delegation, and one to enable after a delegation. If you did not call the function to enable auto redeleagtion upon delegation, you can call an additional function called enable auto redelegation. Either way you enable, you need to provide a fee. This fee gets provided to any user who calls autoRedelegate on your belhaf. Basically, the idea is that due to async nature of the F1 rewards scheme and lack of an onBlock event, autoRedelegation becomes a feature that people might want as they might want to \say{set it and forget it} (i.e. they want their stake to compound automatically. In this design, people are incentivized to send transactions to autoRedlegate rewards for users based on the fee. So for example, if the fee is 2\% for a particular user, people would wait until enough rewards have been accumulated for an account such that 2\% of that value is an appropriate reward for me to send this transaction. Batching service should be provided, to make these transactions even more economical. 

\begin{figure}[ht]
\centering
\includegraphics{sequence-diagrams/auto-redelegate.tikz}
\caption{Auto-redelegation feature}
\label{fig:redelegation}
\end{figure}
\clearpage

\subsection{Delegation Transfer}

Figure \ref{fig:delegation_transfer} demonstrates a delegation transfer feature. This is the idea that if I delegated to a pool, I should be able to transfer my delegation to another pool, with minimal overhead. We can't make the transfers instantaneous, as that creates an attack vector in unity (if you can instantaneously transfer, then you can compute the next delta parameter by changing the amount of stake, to preferentially win blocks!). Therefore we need to impose some delay between when you send a transfer delegation request, and when it takes effect. This delay is smaller than the undelegation delay. This transfer is a two step process (just like auto-delegation and private unstake, and undelegation. Unlike auto redelegation, there is no fee for people to finalize transfers since the pools have an incentive to finalize transfers. We recommend the implementation of a batch protocol to make it even more efficient to finalize a whole bunch of transfer in one shot. Furthermore, the user can wait the transfer lockout period and send the finalize transfer transaction themselves. 

\begin{figure}[ht]
\centering
\includegraphics{sequence-diagrams/delegation-transfer.tikz}
\caption{Delegation transfer feature}
\label{fig:delegation_transfer}
\end{figure}
\clearpage

\subsection{Withdrawal}

Figure \ref{fig:withdrawal} demonstrates a withdrawal. See section \ref{f1-rewards} for details on the function of the coinbase contract; it get's emptied out every time any action is called that changes the stake distribution in the pool (delegate, undelegate, withdraw or transfer delegation actions). In the withdrawal scenario, the pool registry has kept track of how much coins it's already withdrawn from the PoolCoinbase based on delegators' previous actions; it further empties out the pool coinbase contract account by calling the transfer function. It then uses the F1 algorithm (\S\ref{fig:delegation_scheme}) to compute rewards owed to the delegator, and sends it to the delegator.  

\begin{figure}[ht]
\centering
\includegraphics{sequence-diagrams/withdrawal.tikz}
\caption{Delegation rewards withdrawal}
\label{fig:withdrawal}
\end{figure}
\clearpage